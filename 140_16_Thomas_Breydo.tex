\documentclass{amsart}
\usepackage{thomas} % my style file, https://git.io/thomas.sty


\newcommand{\pagenum}{140}
\newcommand{\probnum}{16}

\title{\pagenum.\probnum}
\author{Thomas\ Breydo}
\date{\today}

\begin{document}

\maketitle

\textbf{Problem:} Suppose $V$ is a complex vector space,
$T\in\SL(V),$ and the matrix of $T$ with respect
to a basis of $V$ contains only real entries. Show that if $\lambda$
is an eigenvalue of $T,$ then so is $\overline\lambda.$

\vspace{0.5in}

\newcommand{\basis}{v_1,\ldots,v_n}

\textbf{Solution:} Let $\basis$ be a basis of $V$ for which
\begin{align*}
    \SM(T)=\begin{pmatrix}
        A_{11} & \cdots & A_{1n} \\
        \vdots & \ddots &\vdots \\
        A_{n1}    &\cdots & A_{nn}
    \end{pmatrix}
\end{align*}
where $A_{ij}\in\R.$ Next, suppose that $v$ is an eigenvector of $\lambda$ and that
\begin{align*}
    \SM(v)=\begin{pmatrix}
        c_1\\
        \vdots\\
        c_n\\
    \end{pmatrix}
\end{align*}
where $c_i\in\C.$ It follows that $Tv=\lambda v.$ Or, equivalently,
\begin{align*}
    \begin{pmatrix}
        A_{11} & \cdots & A_{1n} \\
        \vdots & \ddots &\vdots \\
        A_{n1}    &\cdots & A_{nn}
    \end{pmatrix}
    \begin{pmatrix}
        c_1\\
        \vdots\\
        c_n\\
    \end{pmatrix}
    =\begin{pmatrix}
        \lambda c_1\\
        \vdots\\
        \lambda c_n\\
    \end{pmatrix}.
\end{align*}
Thus,
\begin{align*}
    A_{11}c_1+\cdots+A_{1n}c_n&=\lambda c_1\\
    A_{21}c_1+\cdots+A_{2n}c_n&=\lambda c_2\\
    \vdots\\
    A_{n1}c_1+\cdots+A_{nn}c_n&=\lambda c_n
\end{align*}
Taking the complex conjugate on both sides, since $A_{ij}\in\R,$
\begin{align*}
    A_{11}\overline{c_1}+\cdots+A_{1n}\overline{c_n}&=\overline{\lambda}\overline{c_1}\\
    A_{21}\overline{c_1}+\cdots+A_{2n}\overline{c_n}&=\overline{\lambda}\overline{c_2}\\
    \vdots\\
    A_{n1}\overline{c_1}+\cdots+A_{nn}\overline{c_n}&=\overline{\lambda}\overline{c_n}
\end{align*}
It follows that,
\begin{align*}
    \begin{pmatrix}
        A_{11} & \cdots & A_{1n} \\
        \vdots & \ddots &\vdots \\
        A_{n1}    &\cdots & A_{nn}
    \end{pmatrix}
    \begin{pmatrix}
        \overline{c_1}\\
        \vdots\\
        \overline{c_n}\\
    \end{pmatrix}
    =\begin{pmatrix}
        \overline{\lambda}\overline{c_1}\\
        \vdots\\
        \overline{\lambda}\overline{c_n}\\
    \end{pmatrix}.
\end{align*}
Letting
\begin{align*}
    \overline v = \begin{pmatrix}
        \overline{c_1}\\
        \vdots\\
        \overline{c_n}
    \end{pmatrix},
\end{align*}
we see that
\begin{align*}
    T\overline v=\overline\lambda\overline v.
\end{align*}
Since $v\ne0,$ we know that $\overline v\ne 0.$ Thus, $\overline\lambda$
is an eigenvalue of $V$.\qed

\vspace{0.5in}

\textbf{Source code:} \href{https://github.com/thomasbreydo/linalg/blob/main/\pagenum_\probnum_Thomas_Breydo.tex}
{Click here to open the source code on GitHub.}

\end{document}
