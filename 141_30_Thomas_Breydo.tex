\documentclass{amsart}
\usepackage{thomas}

\newcommand{\pagenum}{141}
\newcommand{\probnum}{30}

\title{\pagenum.\probnum}
\author{Thomas\ Breydo}
\date{\today}

\begin{document}

\maketitle

\begin{problem*}
Suppose $T\in\SL(\R^3)$ and $-4,5,$ and $\sqrt 7$
are eigenvalues of $T.$ Prove that there exists $x\in\R^3$
such that $Tx-9x=(-4,5,\sqrt 7).$
\end{problem*}

\vspace{0.5in}

\begin{claim*}
9 is not an eigenvalue of $T.$
\end{claim*}
\begin{proof}
$T$ can have at most $\dim\R^3=3$ eigenvalues by \textit{5.13},
and we already have three distinct eigenvalues: $-4,5,$ and
$\sqrt 7$.
\end{proof}

\begin{claim*}
$T-9I$ is invertible.
\end{claim*}
\begin{proof}
Otherwise, 9 would be an eigenvalue of $T$ by \textit{5.6}.
\end{proof}

\begin{claim*}
    $x=(T-9I)^{-1}\big((-4,5,\sqrt 7)\big)$ works.
\end{claim*}
\begin{proof} Note that
\begin{align*} 
    Tx-9x &= (T-9I)x \\
          &= (-4,5,\sqrt 7)
\end{align*}
as desired.
\end{proof}

\vspace{0.5in}

\begin{note*}
You can view the source code for this solution
\href{https://github.com/thomasbreydo/linalg/blob/main/\pagenum_\probnum_Thomas_Breydo.tex}
{here}.
\end{note*}

\end{document}
