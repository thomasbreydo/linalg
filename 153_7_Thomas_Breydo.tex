\documentclass{amsart}
\usepackage{thomas} % my style file, https://git.io/thomas.sty


\newcommand{\pagenum}{153}
\newcommand{\probnum}{7}

\title{\pagenum.\probnum}
\author{Thomas\ Breydo}

\begin{document}

\maketitle

\begin{problem*}
Suppose $T\in\SL(V).$ Prove that $9$ is an eigenvalue of $T^2$
if and only if $3$ or $-3$ is an eigenvalue of $T.$
\end{problem*}

\vspace{0.5in}


\begin{claim}
\label{impl}
If $\lambda\in\F$ is an eigenvalue of $T,$ then $\lambda^2$ is an
eigenvalue of $T^2.$
\end{claim}
\begin{proof}
Since $\lambda$ is an eigenvalue of $T,$ there exists
a $v\in V$ such that $v\ne 0$ and 
\begin{align*}
    Tv=\lambda v.
\end{align*}
For this $v,$
\begin{align*}
    T^2(v)&=T(Tv)\\
          &=T(\lambda v)\\
          &=\lambda^2 v.
\end{align*}
Thus, $\lambda^2$ is an eigenvalue of $T^2.$
\end{proof}



\begin{claim}
\label{conv}
If $\mu\in\F$ is an eigenvalue of $T^2,$ then $\sqrt\mu$ or $-\sqrt\mu$
is an eigenvalue of $T$ (assuming that $\sqrt\mu\in\F$).
\end{claim}
\begin{proof}
Since $\mu$ is an eigenvalue of $T^2,$ we know that
$T^2-\mu I$ is not injective. Thus, there exists
a $v\in V$ such that $v\ne0$ and
\begin{align*}
    (T^2-\mu I)v=0.
\end{align*}
Since
\begin{align*}
    (T-\sqrt\mu I)(T+\sqrt\mu I)
    &= T^2+T\cdot\sqrt\mu I-\sqrt\mu I\cdot T+(\sqrt\mu I)^2 \\
    &= T^2+\sqrt\mu T-\sqrt\mu T+\left(\sqrt\mu\right)^2 I \\
    &= T^2+\mu I,
\end{align*}
it follows that
\begin{align*}
    (T-\sqrt\mu I)\underbrace{(T+\sqrt\mu I)v}_w=0.
\end{align*}
Consider $w=(T+\sqrt\mu I)v$.

\begin{itemize}
\item If $w=0,$ then $(T+\sqrt\mu I)v=0\implies T+\sqrt\mu I$ is
    not injective. Thus, $-\sqrt\mu$ is an eigenvalue of $T.$

\item If $w\ne0,$ then $(T-\sqrt\mu I)w=0\implies T-\sqrt\mu I$ is
    not injective. Thus, $\sqrt\mu$ is an eigenvalue of $T.$\qedhere
\end{itemize}
\end{proof}

These two claims suffice. It follows from
\textsf{\textbf{Claim \ref{impl}}}  that if $3$ or $-3$ is an
eigenvalue of $T,$ then $9$ is an eigenvalue of $T^2$. It follows from
\textsf{\textbf{Claim \ref{conv}}} that if $9$ is an eigenvalue
of $T^2,$
then $3$ or $-3$ is an eigenvalue $T.$

\vspace{0.5in}

\begin{note*}
You can view the source code for this solution
\href{https://github.com/thomasbreydo/linalg/blob/main/\pagenum_\probnum_Thomas_Breydo.tex}
{here}.
\end{note*}

\end{document}
