\documentclass{amsart}
\usepackage{thomas} % my style file, https://git.io/thomas.sty


\newcommand{\pagenum}{175}
\newcommand{\probnum}{5}

\title{\pagenum.\probnum}
\author{Thomas\ Breydo}

\begin{document}

\maketitle

\begin{problem*}
Suppose $T\in\SL(V)$ is such that
\begin{align*}
    \norm{Tv}\le\norm{v}
\end{align*}
for every $v\in V.$ Prove that $T-\sqrt2I$ is invertible.
\end{problem*}

\vspace{0.5in}

First, we prove the following claim:

\begin{claim*}
If $\lambda>1$ is an eigenvalue of $T$ then there exists a $v\in V$ for which
\begin{align*}
    \norm{Tv}>\norm{v}.
\end{align*}
\end{claim*}
\begin{proof}
Take $v\in V$ to be an eigenvector with eigenvalue $\lambda.$ Notice that
\begin{align*}
    v\ne0 &\implies \norm{v}\ne0
\end{align*}
Multiplying both sides of $\lambda>1$ by $\norm{v},$ we get that
\begin{align*}
    \lambda\norm{v} > \norm{v}.
\end{align*}
Then, since $Tv=\lambda v,$
\begin{align*}
    \norm{Tv} &= \norm{\lambda v} \\
              &= \lambda\norm{v} \\
              &> \norm{v}.\qedhere
\end{align*}
\end{proof}

\begin{claim*}
    $T-\sqrt2I$ is invertible.
\end{claim*}
\begin{proof}
Suppose the contrary, that is not invertible.
Then, $\lambda=\sqrt2$ is an eigenvalue
of $T.$ Since $\lambda>1,$ the previous claim tells us that
there exists a $v\in V$ for which
\begin{align*}
    \norm{Tv}>\norm{v}.
\end{align*}
Thus it is not true that
\begin{align*}
    \norm{Tv}\le\norm{v}
\end{align*}
for every $v\in V,$ which is a contradiction.
So, $T-\sqrt2I$ is not invertible.
\end{proof}

\vspace{0.5in}

\begin{note*}
You can view the source code for this solution
\href{https://github.com/thomasbreydo/linalg/blob/main/\pagenum_\probnum_Thomas_Breydo.tex}
{here}.
\end{note*}

\end{document}
