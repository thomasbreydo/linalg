% code formatting from https://www.overleaf.com/learn/latex/Code_listing 
\documentclass{amsart}
\usepackage{thomas} % my style file, https://git.io/thomas.sty
\usepackage{listings}
\usepackage{xcolor}

\definecolor{codegreen}{rgb}{0,0.6,0}
\definecolor{codegray}{rgb}{0.5,0.5,0.5}
\definecolor{codepurple}{rgb}{0.58,0,0.82}
\definecolor{backcolour}{rgb}{1, 1, 1}


\newcommand{\pagenum}{190}
\newcommand{\probnum}{5}

\title{\pagenum.\probnum}
\author{Thomas\ Breydo}

\begin{document}

\maketitle

\begin{problem*}
On $\SP_2(\R),$ consider the inner product given by
\begin{align*}
    \iprod{p,q}=\int_0^1\!p(x)q(x)\;\mathrm dx.
\end{align*}
Apply the Gram--Schmidt Procedure to the basis $1,x,x^2$ to produce an
orthonormal basis of $\SP_2(\R).$
\end{problem*}

\vspace{0.5in}

\begin{claim*}
    We can write code that applies the Procedure.
\end{claim*}
\begin{proof}
Proof by construction, voila:

\lstdefinestyle{mystyle}{
    backgroundcolor=\color{backcolour},
    commentstyle=\color{codegreen},
    keywordstyle=\color{magenta},
    numberstyle=\tiny\color{codegray},
    stringstyle=\color{codepurple},
    basicstyle=\ttfamily\footnotesize,
    breakatwhitespace=false,
    breaklines=true,
    captionpos=b,
    keepspaces=true,
    numbers=left,
    numbersep=5pt,
    showspaces=false,
    showstringspaces=false,
    showtabs=false,
    tabsize=2
}

\lstset{style=mystyle}

\begin{lstlisting}[language=Python]
def gram_schmidt(vectors: list[Vector]) -> list[Vector]:
    output = []
    for v in vectors:
        next_e = sum([-e.scaled(v * e) for e in output], start=v)
        output.append(next_e.normalized())
    return output
\end{lstlisting}
\end{proof}

\begin{note*}
If you're curious about how I implemented the \texttt{Vector} objects
so that they can be scaled, multiplied, etc, check out
my code \href{https://github.com/thomasbreydo/pylinearalg/blob/main/pylinearalg/vector.py}
{here}.
\end{note*}

\vspace{0.5in}

\begin{claim*}
We can write a function that computes
\begin{align*}
    \int_0^1{\!p(x)q(x)\;\mathrm dx}.
\end{align*}
\end{claim*}

\begin{proof}
Suppose
\begin{align*}
    p(x) &= a_0+a_1x+\dots+a_nx^n \\
    q(x) &= b_0+b_1x+\dots+b_nx^n.
\end{align*}
Then,
\begin{align*}
    \int_0^1{\!p(x)q(x)\;\mathrm dx} &=
    \int_0^1{\!(a_0+a_1x+\dots+a_nx^n)
    (b_0+b_1x+\dots+b_nx^n)\;\mathrm dx}\\
    &= 
    \int_0^1{\! a_0b_0 \;\mathrm dx}
    +
    \int_0^1{\! a_0b_1x \;\mathrm dx}
    +\dots+
    \int_0^1{\! a_0b_nx^n \;\mathrm dx}
    \\
    &+
    \int_0^1{\! a_1b_0 \;\mathrm dx}
    +
    \int_0^1{\! a_1b_1x \;\mathrm dx}
    +\dots+
    \int_0^1{\! a_1b_nx^n \;\mathrm dx}
    \\
    &\qquad\qquad\qquad\qquad\qquad\vdots\\
    &+
    \int_0^1{\! a_nb_0 \;\mathrm dx}
    +
    \int_0^1{\! a_nb_1x \;\mathrm dx}
    +\dots+
    \int_0^1{\! a_nb_nx^n \;\mathrm dx}.
\end{align*}
Each monomial is easy to integrate, since
\begin{align*}
    \int_0^1{\! a_ix^ib_jx^j \;\mathrm dx}
    &= a_ib_j\int_0^1{\! x^{i+j} \;\mathrm dx} \\
    &= \left.\frac{a_ib_j}{i+j+1}x^{i+j+1}\right|^1_0\\
    &=\frac{a_ib_j}{i+j+1}.
\end{align*}
You can see this fact used on line 7 below.

\lstset{style=mystyle}
\begin{lstlisting}[language=Python]
def inner_product(p, q):
    """Evaluate the definite integral from 0 to 1 of pq."""
    total = 0
    for exp1, coef1 in enumerate(p.components):
        for exp2, coef2 in enumerate(q.components):
            # integral of (coef1)x^exp1 + (coef2)x^exp2 is:
            total += coef1 * coef2 / (exp1 + exp2 + 1)
    return total
\end{lstlisting}
\end{proof}

\begin{claim*}
The orthogonal basis is
\begin{align*}
    e_1 &= 1 \\
    e_2 &\approx -1.73+3.456x \\
    e_3 &\approx 2.23-13.42x+13.42x^2.
\end{align*}
\end{claim*}
\begin{proof}
The following snippet of code produces the desired result.
You can confirm the orthagonality of these polynomials
by looking at \href{https://www.desmos.com/calculator/3z6r9ky3wr}
{this} Desmos project.

\lstset{style=mystyle}
\begin{lstlisting}[language=Python]
basis = [BasisVector("1"), BasisVector("x"), BasisVector("x^2")]
V = VectorSpace(basis, inner_product)

u = Vector([1, 0, 0], space=V)
v = Vector([0, 1, 0], space=V)
w = Vector([0, 0, 1], space=V)

for vec in gram_schmidt([u, v, w]):
    print(vec.normalized())
\end{lstlisting}
\end{proof}


\begin{note*}
You can view the source code for this solution
\href{https://github.com/thomasbreydo/linalg/blob/main/\pagenum_\probnum_Thomas_Breydo.tex}
{here}.
\end{note*}

\end{document}
