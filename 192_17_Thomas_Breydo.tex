\documentclass{amsart}
\usepackage{thomas} % my style file, https://git.io/thomas.sty


\newcommand{\pagenum}{190}
\newcommand{\probnum}{17}

\title{\pagenum.\probnum}
\author{Thomas\ Breydo}

\begin{document}

\maketitle

\begin{problem*}
For $u\in V,$ let $\Phi u$ denote the linear functional $V$ defined
by
\begin{align*}
    (\Phi u)(v)=\iprod{v,u}
\end{align*}
for $v\in V.$

\begin{enumerate}[label=(\alph*),topsep=2ex,itemsep=2ex]
    \item Show that if $\F=\R,$ then $\Phi$ is a linear map from $V$
        to $V'.$ (Recall from Section 3.F that $V'=\SL(V,\F)$
        and that $V'$ is called the dual space of $V.$)

    \item Show that if $\F=\C$ and $V\ne\{0\}$, then $\Phi$ is not
        a linear map.

    \item Show that $\Phi$ is injective.

    \item Suppose $\F=\R$ and $V$ is finite-dimensional. Use parts
        (a) and (c) and a dimension-counting argument (but without
        using 6.42) to show that $\Phi$ is an isomorphism from
        $V$ onto $V'.$
\end{enumerate}
\end{problem*}

\begin{note*}
    Part (d) gives an alternative proof of the Reisz Representation
    Theorem (6.42) when $\F=\R.$ Part (d) also gives a natural
    isomorphism (meaning that it does not depend on a choice of
    basis) from a finite-dimensional real inner product space
    onto its dual space.
\end{note*}

\vspace{0.5in}

\begin{claim}
If $\F=\R,$ then $\Phi$ is a linear map from $V$ to $V'.$
\end{claim}
\begin{proof}
Suppose $u_1,u_2\in V.$ Then, the functional $\Phi(u_1+u_2)$ sends
$v\in V$ to
\begin{align*}
    \big(\Phi(u_1+u_2)\big)(v) &= \iprod{v,u_1+u_2} \\
                     &= \iprod{v,u_1}+\iprod{v,u_2} \\
                     &= (\Phi u_1)(v)+(\Phi u_2)(v).
\end{align*}
Thus, $\Phi$ is additive.

Next, suppose $\lambda\in\F$ and
$u\in V.$ Then, the functional $\Phi(\lambda u)$ sends
sends $v\in V$ to
\begin{align*}
    \big(\Phi(\lambda u)\big)(v) &= \iprod{v, \lambda u} \\
                                 &= \overline\lambda\iprod{v, u} \\
                                 &= \lambda\iprod{v,u} &&(\F=\R) \\
                                 &= \lambda(\Phi{u})(v).
\end{align*}
Thus, $\Phi$ is homogenous. Since it is both additive and homogenous,
it is linear.
\end{proof}

\vspace{0.5in}

\begin{claim}
If $\F=\C$ and $V\ne\{0\},$ then $\Phi$ is not a linear map.
\end{claim}
\begin{proof}
Since $V\ne\{0\},$ there exists a $u\in V$ such that $u\ne 0.$
We will show that
\begin{align*}
\Phi(iu)\ne i\Phi(u),
\end{align*}
and thus $\Phi$ is not a linear map. Namely,
$\Phi(iu)$ and $i\Phi(u)$ send $u$ to two different values:
\begin{align*}
    \big(\Phi(iu)\big)(u) &= \iprod{u,iu} \\
                          &= \overline i\iprod{u,u} \\
                          &= -i\iprod{u,u},
\end{align*}
while
\begin{align*}
    \big(i\Phi(u)\big)(u) &= i\big(\Phi(u)\big)(u) \\
                          &= i\iprod{u,u}.
\end{align*}
These are indeed different because $\iprod{u,u}\ne0$ (since $u\ne 0$).
\end{proof}

\vspace{0.5in}

\begin{claim}
 $\Phi$ is injective.
\end{claim}
\begin{proof}
Suppose $u\in V$ and $\Phi u$ is the zero map. Then,
\begin{align*}
    (\Phi u)(u) = 0.
\end{align*}
Since $(\Phi u)(u)=\iprod{u,u},$ the definiteness of the inner
product implies $u=0.$
Thus, $\Phi$ is injective.
\end{proof}

\vspace{0.5in}

\begin{claim}
    If $\F=\R$ and $V$ is finite-dimensional, then
    $\Phi$ is an isomorphism from $V$ onto $V'.$
\end{claim}
\begin{proof}
By \textsf{\textbf{Claim 1}} and \textsf{\textbf{Claim 3}},
$\Phi$ is an injective linear map from $V$ onto $V'.$ Thus,
\begin{align*}
    \dim V' &= \dim V \\
            &= \dim\vrange\Phi+\dim\vnull\Phi \\
            &= \dim\vrange\Phi.
\end{align*}
But if $\dim\vrange\Phi=\dim V'$, then  $\Phi$ must be surjective.
Since it is both injective and surjective, it is invertible.
Thus, it is an isomorphism.
\end{proof}

\vspace{0.5in}


\begin{note*}
You can view the source code for this solution
\href{https://github.com/thomasbreydo/linalg/blob/main/\pagenum_\probnum_Thomas_Breydo.tex}
{here}.
\end{note*}

\end{document}
