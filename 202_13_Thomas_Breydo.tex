\documentclass{amsart}
\usepackage{thomas} % my style file, https://git.io/thomas.sty


\newcommand{\pagenum}{202}
\newcommand{\probnum}{13}

\title{\pagenum.\probnum}
\author{Thomas\ Breydo}

\begin{document}

\maketitle

\begin{problem*}
Find $p\in\SP_5(\R)$ that makes
\begin{align*}
    cost=\int_{-\pi}^\pi\!\abs{\sin x-p(x)}^2\,\mathrm dx
\end{align*}
as small as possible.
\end{problem*}

\vspace{0.5in}

Consider the vector space $V$ of all continuous functions
from $-\pi$ to $\pi$, with the inner product
\begin{align*}
    \iprod{f,g}=\int_{-\pi}^\pi\! f(x)g(x)\,\mathrm dx.
\end{align*}
Let $U=\SP_5(\R),$ and note that $U$ is a subspace of $V.$
\begin{claim*}
The optimal $p\in U$ is that which minimizes
$\norm{\sin x - p(x)}.$
\end{claim*}
\begin{proof}
Note that
\begin{align*}
    \norm{\sin x - p(x)} &= \sqrt{\iprod{\sin x-p(x),\sin x-p(x)}} \\
                         &= \sqrt{
    \int_{-\pi}^\pi\!(\sin x-p(x))^2\,\mathrm dx
                         } \\
                         &= \sqrt{cost},
\end{align*}
so minimizing $\norm{\sin x - p(x)}$ minimizes the cost.
\end{proof}

\begin{claim*}
The optimal $p\in U$ is $P_U\sin x.$
\end{claim*}
\begin{proof}
    
By \textit{6.56}, we know that for any $p\in U$
\begin{align*}
\norm{\sin x - P_U\sin x}\le \norm{\sin x- p(x)},
\end{align*}
and that $p(x)=P_U\sin x$ is the only $p$ for which
equality holds, minimizing the right-hand side,
and in turn minimzing $cost$.
\end{proof}

\newpage

\begin{claim*}
The optimal $p\in U$ is
\begin{align*}
    p(x)=\frac{- 72765 \pi^{2} + 693 \pi^{4} + 654885}
    {8 \pi^{10}} x^{5}
    &+ \frac{-363825 - 315 \pi^{4} + 39375 \pi^{2}}
    {4 \pi^{8}} x^{3} \\
    &+ \frac{- 16065 \pi^{2}+ 105 \pi^{4} + 155925}{8 \pi^{6}} x
\end{align*}
Click \href{https://www.desmos.com/calculator/dldsdtkuav}{here}
to check out the approximation on Desmos.
\end{claim*}
\begin{proof}
We can apply
the Gram--Schmidt Procedure to $1,x,x^2,x^3,x^4,x^5$
to obtain an orthonormal basis of $U.$
Then, we can compute
\begin{align*}
    p(x) &= P_U\sin x \\
         &= \iprod{e_1,\sin x}e_1+\cdots\iprod{e_6,\sin x}e_6.
\end{align*}

I expanded my \texttt{pylinearalg} Python package to use \texttt{sympy},
which can handle ``symbolic integration'' (instead of floating-point
approximations). You can see the solution script
\href{https://github.com/thomasbreydo/pylinearalg/blob/main/examples/page202_prob13.py}{here}.
\end{proof}

\vspace{0.5in}

\begin{note*}
You can view the source code for this solution
\href{https://github.com/thomasbreydo/linalg/blob/main/\pagenum_\probnum_Thomas_Breydo.tex}
{here}.
\end{note*}

\end{document}
