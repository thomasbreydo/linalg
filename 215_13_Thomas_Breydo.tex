\documentclass{amsart}
\usepackage{thomas} % my style file, https://git.io/thomas.sty


\newcommand{\pagenum}{215}
\newcommand{\probnum}{13}

\title{\pagenum.\probnum}
\author{Thomas\ Breydo}

\begin{document}

\maketitle

\begin{problem*}
Give an example of an operator $T\in\SL(\C^4)$ such that $T$ is
normal but not self-adjoint.
\end{problem*}

\vspace{0.5in}

Define $T$ by $Tv=iv$ for all $v\in\C^4.$
\begin{claim*}
$T^*v=-iv.$
\end{claim*}
\begin{proof}
Fix $u\in\C^4.$ Then, for all $v\in\C^4,$
\begin{align*}
    \iprod{u, T^*v}
    &= \iprod{Tu, v} &&(\textit{definition of $T^*$})\\
    &= \iprod{iu, v} \\
    &= i\iprod{u, v} \\ 
    &= \iprod{u, -iv}
\end{align*}
Thus, $T^*v=-iv$ as desired.
\end{proof}

\begin{claim*}
$T$ is normal.
\end{claim*}
\begin{proof}
For all $v\in\C^4,$
\begin{align*}
    TT^*v &= (i)(-i)v \\
          &= (-i)(i)v \\
          &= T^*Tv.
\end{align*}
Thus, $TT^*=T^*T,$ which means that $T$ is normal.
\end{proof}

\begin{claim*}
$T$ is not self-adjoint.
\end{claim*}
\begin{proof}
    Let $v=(1,1,1,1).$ Notice that
\begin{align*}
    Tv=(i,i,i,i),
\end{align*}
while
\begin{align*}
    T^*v=(-i,-i,-i,-i),
\end{align*}
so $T\ne T^*.$
\end{proof}

\vspace{0.5in}

\begin{note*}
You can view the source code for this solution
\href{https://github.com/thomasbreydo/linalg/blob/main/\pagenum_\probnum_Thomas_Breydo.tex}
{here}.
\end{note*}

\end{document}
