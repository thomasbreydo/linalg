\documentclass{amsart}
\usepackage{thomas} % my style file, https://git.io/thomas.sty


\newcommand{\pagenum}{223}
\newcommand{\probnum}{8}

\title{\pagenum.\probnum}
\author{Thomas\ Breydo}

\begin{document}

\maketitle

\begin{problem*}
Give an example of an operator $T$ on a complex
vector space such that $T^9=T^8$ but $T^2\ne T.$
\end{problem*}

\vspace{0.5in}

The key idea is to construct a $T$ such that $T^2,T^3,\dots,T^9$
are all the zero map, while $T$ itself is not.

Consider the complex vector space $\C$ with basis $1+0i,0+1i.$
Let $T$ be the linear map defined by
\begin{align*}
    T(a+bi)=b+0i
\end{align*}
for all $a+bi\in\C.$

\begin{claim*}
$T^2\equiv 0.$
\end{claim*}
\begin{proof}
For all $a+bi\in\C,$
\begin{align*}
    T^2(a+bi) &= T(b+0i)\\
              &= 0+0i.
\end{align*}
Thus, $T^2$ is the zero map. \end{proof}

\begin{claim*}
$T^9=T^8.$
\end{claim*}
\begin{proof}
It follows from the previous claim that $T^8\equiv 0$ and $T^9\equiv 0$
, so $T^8=T^9.$
\end{proof}

\begin{claim*}
$T^2\ne T.$
\end{claim*}
\begin{proof}
Since $$T(1+1i)=1+0i$$ while $$T^2(1+1i)=0+0i,$$ the two maps
are different.
\end{proof}
\vspace{0.5in}

\begin{note*}
You can view the source code for this solution
\href{https://github.com/thomasbreydo/linalg/blob/main/\pagenum_\probnum_Thomas_Breydo.tex}
{here}.
\end{note*}

\end{document}
