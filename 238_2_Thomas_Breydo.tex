\documentclass{amsart}
\usepackage{thomas} % my style file, https://git.io/thomas.sty


\newcommand{\pagenum}{238}
\newcommand{\probnum}{2}

\title{\pagenum.\probnum}
\author{Thomas\ Breydo}

\begin{document}

\maketitle

\begin{problem*}
Give an example of $T\in\SL(\C^2)$ such that 0 is the only eigenvalue
of $T$ and the singular values are $5,0.$
\end{problem*}

\vspace{0.5in}

Suppose
\begin{align*}
    \SM(T)=\begin{pmatrix}
        0 & a \\
        b & 0
    \end{pmatrix}.
\end{align*}
Since the standard basis is orthonormal,
\begin{align*}
    \SM(T^*)=\begin{pmatrix}
        0 & b \\
        a & 0
    \end{pmatrix}.
\end{align*}
Thus,
\begin{align*}
    \SM(T^* T) &= \SM(T^*)\SM(T) \\
               &= \begin{pmatrix}
                   0 & b \\
                   a & 0
               \end{pmatrix}
               \begin{pmatrix}
                   0 & a \\
                   b & 0
               \end{pmatrix} \\
               &= \begin{pmatrix}
                   b^2 & 0 \\
                   0 & a^2
               \end{pmatrix}
\end{align*}
Since we need $T$ to have singular values $0$ and $5,$ we need $T^*T$
to have eigenvalues $0$ and $25.$ If we let $b=0$ and $a=5,$ we get
\begin{align*}
    \SM(T)=\begin{pmatrix}
        0 & 5 \\
        0 & 0
    \end{pmatrix}.
\end{align*}
Not sure how to prove that $0$ is the only eigenvalue of
$T(z_1,z_2)=(5z_2,0).$ Clearly it \textit{is} an eigenvalue,
though.

\vspace{0.5in}

\begin{note*}
You can view the source code for this solution
\href{https://github.com/thomasbreydo/linalg/blob/main/\pagenum_\probnum_Thomas_Breydo.tex}
{here}.
\end{note*}

\end{document}
