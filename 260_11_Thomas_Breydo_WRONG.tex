\documentclass{amsart}
\usepackage{thomas} % my style file, https://git.io/thomas.sty


\newcommand{\pagenum}{260}
\newcommand{\probnum}{11}

\title{\pagenum.\probnum}
\author{Thomas\ Breydo}

\begin{document}

\maketitle

\begin{problem*}
Suppose $T\in\SL(V)$ and $\lambda\in\F.$ Prove that for
every basis of $V$ with respect to which $T$ has
an upper-triangular matrix, the number of times
that $\lambda$ appears on the matrix of $T$ equals
the multiplicity of $\lambda$ as an eigenvalue
of $T.$
\end{problem*}

\vspace{0.5in}

\begin{claim}
For $S\in\SL(V),$ if $\SM(S)$ upper-triangular then it has
$\dim\vnull S$ zeros on its diagonal.
\end{claim}
\begin{proof}
Supppose $\SM(S)$ has $d$ zeros on its diagonal. Then, its
rank is $n-d,$ since the $d$ columns with zeros on the diagonal
are in the span of the columnns that precede them.
\end{proof}

\begin{claim}
Suppose
\begin{align*}
\SM(S)=\begin{pmatrix}
    a_1 & & * \\
        & \ddots &  \\
    0 & & a_n
\end{pmatrix}.
\end{align*}
Then, for $k\ge 1,$
\begin{align*}
    \SM\big(S^k\big)=\begin{pmatrix}
        {a_1}^k & & * \\
            & \ddots &  \\
        0 & & {a_n}^k
    \end{pmatrix}.
\end{align*}
\end{claim}

\begin{proof}
Induct on $k.$ The base case is trivial. Now, suppose
\begin{align*}
    \SM\big(S^k\big)=\begin{pmatrix}
        {a_1}^k & & * \\
            & \ddots &  \\
        0 & & {a_n}^k
    \end{pmatrix}.
\end{align*}
Then,
\begin{align*}
    \SM\big(S^{k+1}\big) &= \SM\big(S^k\big)\SM(S) \\
                 &= \begin{pmatrix}
        {a_1}^k & & * \\
            & \ddots &  \\
        0 & & {a_n}^k
    \end{pmatrix}
    \begin{pmatrix}
        a_1 & & * \\
            & \ddots &  \\
        0 & & a_n
    \end{pmatrix}.
\end{align*}
The $i^\text{th}$ diagonal entry of this product will be
\begin{align*}
\begin{pmatrix}
    0 & \cdots & 0 & {a_i}^k & * & \cdots & *
\end{pmatrix}
\begin{pmatrix}
    * \\
    \vdots \\
    * \\
    a_i \\
    0 \\
    \vdots \\
    0
\end{pmatrix} = {a_i}^{k+1}.
\end{align*}
Since $\SM\big(S^{k+1}\big)$ will also be upper-triangular,
\begin{align*}
    \SM\big(S^{k+1}\big)=\begin{pmatrix}
        {a_1}^{k+1} & & * \\
            & \ddots & \\
        0 & & {a_n}^{k+1}
    \end{pmatrix}.
\end{align*}
\end{proof}


\begin{claim}
$\SM(T-\lambda I)$ is upper-triangular and has $d$
zeros on its diagonal.
\end{claim}
\begin{proof}
Note,
\begin{align*}
    \SM(T-\lambda I) &= \SM(T)-\lambda\SM(I) \\
                     &= \begin{pmatrix}
                         a_{1} & & * \\
                               & \ddots & \\
                         0 & & a_{n}
                     \end{pmatrix}
                     -
                     \begin{pmatrix}
                         \lambda & & 0 \\
                                 & \ddots & \\
                         0 & & \lambda
                     \end{pmatrix} \\
                     &= \begin{pmatrix}
                         a_{1}-\lambda & & * \\
                                       & \ddots & \\
                         0 & & a_{n}-\lambda
                     \end{pmatrix}.
\end{align*}
Since $a_i-\lambda=0$ where $a_i=\lambda,$ we see that $\SM(T-\lambda I)$
has $d$ zeros on its diagonal.
\end{proof}

\begin{claim}
$\SM\big((T-\lambda I)^{\dim V}\big)$ is upper-triangular
and has $d$ zeros on its diagonal.
\end{claim}
\begin{proof}
By Claim 3, we have that $\SM(T-\lambda I)$ is upper-triangular
and has $d$ zeros on its diagonal. It follows from Claim 2
that $\SM\big((T-\lambda I)^{\dim V}\big)$ is upper-triangular
and has $d$ zeros on its diagonal.
\end{proof}

Finally, we will show that a matrix with $d$ zeros has
a null space of dimension $d.$

\vspace{0.5in}

\begin{note*}
You can view the source code for this solution
\href{https://github.com/thomasbreydo/linalg/blob/main/\pagenum_\probnum_Thomas_Breydo.tex}
{here}.
\end{note*}

\end{document}
