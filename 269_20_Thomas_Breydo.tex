\documentclass{amsart}
\usepackage{thomas} % my style file, https://git.io/thomas.sty

\newcommand{\restrict}[2]{\left.#1\right|_{#2}}

\newcommand{\pagenum}{269}
\newcommand{\probnum}{20}

\title{\pagenum.\probnum}
\author{Thomas\ Breydo}

\begin{document}

\maketitle

\begin{problem*}
Suppose $V$ is a complex vector space and $V_1,\dots,V_m$ are
nonzero subspaces of $V$ such that $V=V_1\oplus\dots\oplus V_m.$
Suppose $T\in\SL(V)$ and each $V_j$ is invariant under $T.$
For each $j,$ let $p_j$ denote the characteristic polynomial
of $\restrict{T}{V_j}$ Prove that the characteristic
polynomial of $T$ equals $p_1\cdots p_m.$
\end{problem*}

\vspace{0.5in}

\begin{claim*}
Every eigenvalue of $\restrict{T}{V_j}$ is also an eigenvalue
of $T.$
\end{claim*}
\begin{proof}
Suppose $\lambda$ is an eigenvalue of $\restrict{T}{V_j},$
with corresponding generalized eigenvector $v\in V_j.$
Then,
\begin{align*}
\left(\restrict{T}{V_j}-\lambda I\right)^{\dim V_j}v=0.
\end{align*}
Since $\dim V_j<\dim V$ and $T$ is invariant on $V_j,$
\begin{align*}
\left(T-\lambda I\right)^{\dim V}v=0.
\end{align*}
Thus, $\lambda$ must be one of the eigenvalues of $T$.
\end{proof}

\vspace{\baselineskip}

Suppose the eigenvalues of $T$ are $\lambda_1,\dots,\lambda_k.$

\begin{claim*}
The sum of the multiplicities of some eigenvalue $\lambda$ over all
$\restrict{T}{V_1},\dots,\restrict{T}{V_m}$ is equal to the
multiplicity of eigenvalue $\lambda$ of $T.$ In other words,
\begin{align*}
    \dim G(\lambda,T)
            &= \dim G(\lambda,\restrict{T}{V_1})
            +\cdots+
            \dim G(\lambda,\restrict{T}{V_m}).
\end{align*}

\end{claim*}
\begin{proof}
By \textit{8.21},
\begin{align*}
    V_j &= G(\lambda_1,\restrict{T}{V_j})\oplus\cdots\oplus G(\lambda_k,
    \restrict{T}{V_j}),
\end{align*}
since the eigenvalues of $V_j$ are contained within
$\lambda_1,\dots,\lambda_k$ We use this to go from line
two to line three below:
\begin{align*}
    G(\lambda_1,T)\oplus\cdots\oplus G(\lambda_k,T)
      &= V \\
      &= \bigoplus_{j=1}^m{V_j} \\
      &= \bigoplus_{j=1}^m{\left(
          G(\lambda_1,\restrict{T}{V_j})\oplus\cdots\oplus G(\lambda_k,
    \restrict{T}{V_j})\right)} \\
      &= \left(\bigoplus_{j=1}^{m}G(\lambda_1,\restrict{T}{V_j})\right)
      \oplus\cdots\oplus
      \left(\bigoplus_{j=1}^{m}G(\lambda_k,\restrict{T}{V_j})\right).
\end{align*}
We gather that for each $\lambda,$
\begin{align*}
    G(\lambda,T) &= \bigoplus_{j=1}^m G(\lambda,\restrict{T}{V_j}) \\
                   &= G(\lambda,\restrict{T}{V_1})\oplus
                   \cdots\oplus G(\lambda,\restrict{T}{V_m}),
\end{align*}
and thus
\begin{align*}
    \dim G(\lambda,T) &= \dim\bigg(
        G(\lambda,\restrict{T}{V_1})
        \oplus\cdots\oplus G(\lambda,\restrict{T}{V_m})
    \bigg)\\
                        &= \dim G(\lambda,\restrict{T}{V_1})
                        +\cdots+
                        \dim G(\lambda,\restrict{T}{V_m}).\qedhere
\end{align*}
\end{proof}

\vspace{0.5in}

Suppose the characteristic polynomial of $T$ is $q.$
\begin{claim*}
    $p_1\cdots p_m=q.$
\end{claim*}
\begin{proof}
\begin{align*}
    p_1\cdots p_m &= \prod_{j=1}^m{p_j} \\
                  &= \prod_{j=1}^m{
                      \left((z-\lambda_1)^{\dim G(\lambda_1,\restrict{T}{V_j})}\right)
                  \cdots
              \left((z-\lambda_k)^{\dim G(\lambda_k,\restrict{T}{V_j})}\right)
          }\\
          &= \left(\prod_{j=1}^m
                      (z-\lambda_1)^{\dim G(\lambda_1,\restrict{T}{V_j})}
                      \right)
                      \cdots
                      \left(\prod_{j=1}^m
                      (z-\lambda_k)^{\dim G(\lambda_k,\restrict{T}{V_j})}
                      \right)\\
          &= \left((z-\lambda_1)
          ^{\sum_{j=1}^m\dim G(\lambda_1,\restrict{T}{V_j})}\right)
          \cdots
          \left((z-\lambda_1)
          ^{\sum_{j=1}^m\dim G(\lambda_k,\restrict{T}{V_j})}\right)\\
          &= \left((z-\lambda_1)^{\dim{G(\lambda_1,T)}}\right)
          \cdots
          \left((z-\lambda_k)^{\dim{G(\lambda_k,T)}}\right)\\
          &= q
\end{align*}
\end{proof}

\begin{note*}
You can view the source code for this solution
\href{https://github.com/thomasbreydo/linalg/blob/main/\pagenum_\probnum_Thomas_Breydo.tex}
{here}.
\end{note*}

\end{document}
