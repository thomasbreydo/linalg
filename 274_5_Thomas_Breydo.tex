\documentclass{amsart}
\usepackage{thomas} % my style file, https://git.io/thomas.sty


\newcommand{\pagenum}{274}
\newcommand{\probnum}{5}

\title{\pagenum.\probnum}
\author{Thomas\ Breydo}

\begin{document}

\maketitle

\begin{problem*}
Suppose $T\in\SL(V)$ and $v_1,\dots,v_n$ is a basis of $V$
that is a Jordan basis for $T.$ Describe the matrix
of $T^2$ with respect to this basis.
\end{problem*}

\vspace{0.5in}

Since $v_1,\dots,v_n$ is a Jordan basis for $T,$

\begin{align*}
    \SM(T) &= \begin{pmatrix}
        A_1 & & 0 \\
            & \ddots & \\
        0 & & A_p
    \end{pmatrix},
\end{align*}

where

\begin{align*}
    A_j &= \begin{pmatrix}
        \lambda_j & 1 & & 0 \\
                  & \ddots & \ddots & \\
                  &       & \ddots & 1 \\
        0 & & & \lambda_j
    \end{pmatrix},
\end{align*}

\begin{claim*}
    \begin{align*}
    \SM(T^2) &= \begin{pmatrix}
        A_1^2 & & 0 \\
            & \ddots & \\
        0 & & A_p^2
    \end{pmatrix},
    \end{align*}
\end{claim*}
\begin{proof}
If you do the matrix multiplication $\SM(T)\cdot\SM(T),$ you will get
the desired result. Intuitively, if $T$ acts independently on
the $p$ subspaces, then $T^2$ will act exactly as described by
$A_1^2,\dots,A_p^2.$
\end{proof}


As a result of the claim above, we see that

\begin{align*}
    \SM(T^2) &= \begin{pmatrix}
        B_1 & & 0 \\
            & \ddots & \\
        0 & & B_p
    \end{pmatrix},
\end{align*}

where

\begin{align*}
    B_j &= A_j^2 \\
        &= \begin{pmatrix}
            \lambda_j^2 & 2\lambda _j & 1 & & 0 \\
                        & \ddots & \ddots & \ddots & \\
                        & & \ddots & \ddots & 1 \\
                        &       & & \ddots & 2\lambda_j \\
            0 & & & & \lambda_j
    \end{pmatrix}.
\end{align*}


\vspace{0.5in}

\begin{note*}
You can view the source code for this solution
\href{https://github.com/thomasbreydo/linalg/blob/main/\pagenum_\probnum_Thomas_Breydo.tex}
{here}.
\end{note*}

\end{document}
