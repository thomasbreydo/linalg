\documentclass{amsart}
\usepackage{thomas} % my style file, https://git.io/thomas.sty


\newcommand{\pagenum}{285}
\newcommand{\probnum}{7}

\title{\pagenum.\probnum}
\author{Thomas\ Breydo}

\begin{document}

\maketitle

\begin{problem*}
Suppose $V$ is a real vector space and $N\in\SL(V).$ Prove that
$N_C$ is nilpotent if and only if $N$ is nilpotent.
\end{problem*}

\vspace{0.5in}

\newcommand{\dimv}{k}

Suppose $k=\dim V.$

\begin{claim*}
$N^{\dimv}=0 \then (N_C)^{\dimv}=0$
\end{claim*}
\begin{proof}
Suppose $N^{\dimv}=0.$ Then, for any $u+iv\in V_C,$
\begin{align*}
    (N_C)^\dimv(u+iv) &= N^\dimv u+iN^\dimv v && \text{(by 9.9)} \\
              &= 0,
\end{align*}
and thus $(N_C)^k=0.$
\end{proof}

\begin{claim*}
$(N_C)^{\dimv}=0 \then N^{\dimv}=0$
\end{claim*}
\begin{proof}
Suppose $(N_C)^\dimv=0.$ Then, for any $u\in V,$
\begin{align*}
    N^ku &= N^ku + i\cdot 0 \\
         &= N^ku + iN^k0 \\
         &= (N_C)^\dimv(u+i\cdot 0) && \text{(by 9.9)} \\
         &= 0,
\end{align*}
and thus $N^k=0.$
\end{proof}

Combining the two claims above, we get that
\begin{align*}
    N^k = 0 \quad\iff\quad (N_C)^k=0.
\end{align*}

\vspace{0.5in}

\begin{note*}
You can view the source code for this solution
\href{https://github.com/thomasbreydo/linalg/blob/main/\pagenum_\probnum_Thomas_Breydo.tex}
{here}.
\end{note*}

\end{document}
