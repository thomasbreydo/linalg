\documentclass{amsart}
\usepackage{footnote}
\usepackage{thomas} % my style file, https://git.io/thomas.sty


\newcommand{\pagenum}{286}
\newcommand{\probnum}{18}

\title{\pagenum.\probnum}
\author{Thomas\ Breydo}

\begin{document}

\maketitle

\begin{problem*}
Suppose $V$ is a real vector space and $T\in\SL(V).$ Prove that the
following are equivalent:
\begin{enumerate}[label=(\alph*)]
    \item All the eigenvalues of $T_\C$ are real.
    \item There exists a basis of $V$ with respect to which
        $T$ has an upper-triangular matrix.
    \item There exists a basis of $V$ consisting of generalized
        eigenvectors.
\end{enumerate}
\end{problem*}

\vspace{0.5in}

b->a

a->c


\begin{claim*}
(c) implies (b)
\end{claim*}
\begin{proof}
Suppose there exists a basis of $V$ consisting of generalized
eigenvectors. By \textit{8.13}, each vector corresponds to just one eigenvalue.
Reorder the vectors so that all vectors with eigenvalue $\lambda_1$
come first, then $\lambda_2,$ and so on. Within each $\lambda$ group,
sort the vectors $v_i$ in increasing value of $k_i,$ where
\begin{align*}
    k_i\coloneqq\text{smallest }m\text{ for which }(T-\lambda_{v_i} I)^m v_i=0.
\end{align*}

Suppose that the reordered basis is $v_1,\ldots,v_n.$
We will now show that the matrix of $T$ with respect thomas
$v_1,\ldots,v_n$ is upper-triangular. Suppose $v_i$ has eigenvalue
$\lambda$ and $k_i=k,$ and that
\begin{align*}
    Tv_i=a_1v_1+\cdots+a_nv_n.
\end{align*}
We must show that $a_{i+1}=\cdots=a_n=0.$ Since $v_i$ is a generalized
eigenvector with eigenvalue $\lambda,$
\begin{align*}
    0
    &= (T-\lambda I)^{k_i+1}{v_i} \\
    &= (T-\lambda I)^{k_i}(Tv_i-\lambda v_i) \\
    &= (T-\lambda I)^{k_i}\left(a_1v_1+\cdots+a_nv_n-\lambda v_i\right)\\
    &= \sum_{j\ne i}{(T-\lambda I)^{k_i}(a_jv_j)}
        &&\text{see note\footnotemark}\\
    &= \sum_{j\ne i}{a_j(T-\lambda I)^{k_i}v_j}
\end{align*}
\footnotetext{Note,
$(T-\lambda I)^{k_i}(a_iv_i)=(T-\lambda I)^{k_i}(-\lambda v_i)=0$}
But $v_1,\ldots,v_n$ is linearly independent, and so is the list

Since $v_1,\dots,v_n$ is a basis of $V$ with respect to which
$T$ has an upper-triangular matrix, (c) implies (b).
\end{proof}

\vspace{0.5in}

\begin{note*}
You can view the source code for this solution
\href{https://github.com/thomasbreydo/linalg/blob/main/\pagenum_\probnum_Thomas_Breydo.tex}
{here}.
\end{note*}

\end{document}
