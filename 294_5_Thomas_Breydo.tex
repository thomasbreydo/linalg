\documentclass{amsart}
\usepackage{thomas} % my style file, https://git.io/thomas.sty


\newcommand{\pagenum}{294}
\newcommand{\probnum}{5}

\title{\pagenum.\probnum}
\author{Thomas\ Breydo}

\begin{document}

\maketitle

\begin{problem*}
Prove the Real Spectral Theorem via complexification and the
Complex Spectral Theorem.
\end{problem*}

\vspace{0.5in}

Suppose $V$ is a real inner product space, and $T\in\SL(V)$
is self-adjoint. To prove the Real Spectral Theorem,
it suffices to show that $T$ has an eigenvalue. From there,
the proof is identical to our original proof.

Define the
complex inner product on $V_\C$ to be
\begin{align*}
    \iprod{u+iv,x+iy} = \iprod{u,x}+\iprod{v,y}+
    (\iprod{v,x}-\iprod{u,y})i
\end{align*}
for all $u,v,x,y\in V.$ Then, by the previous problem, \textit{9.B.4},
we have that $T_\C$ is self-adjoint.

\begin{claim*}
$\iprod{T_\C(u+iv),u+iv}\in\R$ for all $u,v\in V.$
\end{claim*}
\begin{proof}
We begin with a lemma: if $z\in\C$ and $\overline z=z$ then $z\in\R.$\footnote{
Proof: let $z=a+bi$ for $a,b\in\R.$ Since $\overline z=z,$
we have $a-bi=a+bi,$ so $-b=b,$ so $b=0.$ Thus, $z\in\R.$}
Next, note that
\begin{align*}
    \overline{\iprod{T_\C(u+iv),u+iv}}
    &= \overline{\iprod{Tu+iTv,u+iv}} \\
    &= \overline{\iprod{Tu,u}+\iprod{Tv,v}+(\iprod{Tv,u}-\iprod{Tu,v})i} \\
    &= \overline{\iprod{Tu,u}}+
    \overline{\iprod{Tv,v}}
    +(\overline{\iprod{Tv,u}}-\overline{\iprod{Tu,v}})
    (-i) \\
    &= \iprod{u,Tu}+\iprod{v,Tv}+(\iprod{u,Tv}-\iprod{v,Tu})(-i) \\
    &= \iprod{u,Tu}+\iprod{v,Tv}+(\iprod{v,Tu}-\iprod{u,Tv})i \\
    &= \iprod{u+iv,Tu+iTv} \\
    &= \iprod{u+iv,T_\C(u+iv)} \\
    &= \iprod{T_\C(u+iv), u+iv}.\qquad\qquad(T_C\text{ is self-adjoint})
\end{align*}
Thus, by our lemma, $\iprod{T_\C(u+iv),u+iv}\in\R.$
\end{proof}

\begin{claim*}
If $T_\C(u+iv)=\lambda(u+iv)$ for some $u,v\in V$ and $\lambda\in\C$
then $\lambda\in\R.$
\end{claim*}
\begin{proof}
By the previous claim, the following inner product is real:
\begin{align*}
    \iprod{T_\C(u+iv),u+iv} &= \iprod{\lambda(u+iv),u+iv} \\
                            &= \lambda\iprod{u+iv,u+iv} \\
                            &= \lambda\norm{u+iv}^2.
\end{align*}
Thus, $\lambda\norm{u+iv}^2\in\R.$ Since $\norm{u+iv}^2\in\R,$
we have $\lambda\in\R.$
\end{proof}

\begin{claim*}
$T$ has an eigenvalue.
\end{claim*}
\begin{proof}
By the Complex Spectral Theorem, $T_\C$ has an eigenvalue. Suppose
\begin{align*}
    T_\C(u+iv)=\lambda(u+iv)
\end{align*}
for some $u,v\in V$ and $\lambda\in\C.$
By the previous claim, $\lambda\in\R.$
Furthermore,
\begin{align*}
    Tu+iTv=\lambda u+i\lambda v.
\end{align*}
Since $\lambda\in\R,$ we can compare real/imaginary parts to get
\begin{align*}
Tu=\lambda u\text{ and }Tv=\lambda v.
\end{align*}
Since $u+iv\ne 0,$ at least one of $u,v$ is nonzero. Thus, $\lambda$
is an eigenvalue of $T.$
\end{proof}

\vspace{0.5in}

\begin{note*}
You can view the source code for this solution
\href{https://github.com/thomasbreydo/linalg/blob/main/\pagenum_\probnum_Thomas_Breydo.tex}
{here}.
\end{note*}

\end{document}
