\documentclass{amsart}
\usepackage{thomas} % my style file, https://git.io/thomas.sty


\newcommand{\pagenum}{304}
\newcommand{\probnum}{3}

\title{\pagenum.\probnum}
\author{Thomas\ Breydo}

\begin{document}

\maketitle

\begin{problem*}
Suppose $T\in\SL(V)$ has the same matrix with respect to every
basis of $V.$ Prove that $T$ is a scalar multiple of the
identity operator.
\end{problem*}

\vspace{0.5in}

Let $v_1,\ldots,v_n$ be a basis of $V,$ and suppose $1\le i<j\le n.$
Next, consider the matrix of $T$ with respect to $v_1,\ldots,v_n$:
\begin{align*}
\SM(T,(v_1,\ldots,v_n)) =
    \begin{pmatrix}
        & \vdots & & \vdots & \\
       \hdots & A_{ii} & \hdots & A_{ij} & \hdots \\
                & \vdots & \ddots & \vdots & \\
       \hdots & A_{ji} & \hdots & A_{jj} & \hdots \\
              & \vdots & & \vdots &
    \end{pmatrix}.
\end{align*}
For convenience, let's relabel:
\begin{align*}
\SM(T,(v_1,\ldots,v_n)) =
    \begin{pmatrix}
        & \vdots & & \vdots & \\
       \hdots & a & \hdots & b & \hdots \\
                & \vdots & \ddots & \vdots & \\
       \hdots & c & \hdots & d & \hdots \\
              & \vdots & & \vdots &
    \end{pmatrix}.
\end{align*}
Recall that the $i^\text{th}$ column corresponds to $Tv_i,$ so
\begin{align*}
    Tv_i=\cdots+av_i+\cdots+cv_j+\cdots,
\end{align*}
and the $j^\text{th}$ column corresponds to $Tv_j,$ so
\begin{align*}
    Tv_j=\cdots+bv_i+\cdots+dv_j+\cdots.
\end{align*}

\begin{claim*}
$a=d.$
\end{claim*}
\begin{proof}
If we swap $v_i$ and $v_j$ in the basis, the $i^\text{th}$
column will correspond to
\begin{align*}
Tv_j=\cdots+dv_j+\cdots+bv_i+\cdots,
\end{align*}
while the $j^\text{th}$ column will correspond to
\begin{align*}
Tv_i=\cdots+cv_j+\cdots+av_i+\cdots.
\end{align*}
Thus, under the new basis,
\begin{align*}
    \begin{pmatrix}
        & \vdots & & \vdots & \\
       \hdots & a & \hdots & b & \hdots \\
                & \vdots & \ddots & \vdots & \\
       \hdots & c & \hdots & d & \hdots \\
              & \vdots & & \vdots &
    \end{pmatrix}
    \text{ will become }
    \begin{pmatrix}
        & \vdots & & \vdots & \\
       \hdots & d & \hdots & c & \hdots \\
                & \vdots & \ddots & \vdots & \\
       \hdots & b & \hdots & a & \hdots \\
              & \vdots & & \vdots &
    \end{pmatrix}.
\end{align*}
Since $T$ has the same matrix with respect to
every basis of $V,$ we see that $a=d.$
\end{proof}

\begin{claim*}
$b=c=0.$
\end{claim*}
\begin{proof}
This time, double $v_i$ instead of swapping it with $v_j.$ After
we double $v_i,$ our new basis will be
($\ldots,2v_i,\ldots,v_j,\ldots$). Under this new
basis, the $i^\text{th}$ column will correspond to
\begin{align*}
    T(2v_i)=\cdots+a(2v_i)+\cdots+2c(v_j)+\cdots,
\end{align*}
and the $j^\text{th}$ column will correspond to
\begin{align*}
    T(v_j)=\cdots+\frac12b(2v_i)+\cdots+dv_j+\cdots.
\end{align*}
Thus, under the new basis,
\begin{align*}
    \begin{pmatrix}
        & \vdots & & \vdots & \\
       \hdots & a & \hdots & b & \hdots \\
                & \vdots & \ddots & \vdots & \\
       \hdots & c & \hdots & d & \hdots \\
              & \vdots & & \vdots &
    \end{pmatrix}
    \text{ will become }
    \begin{pmatrix}
        & \vdots & & \vdots & \\
       \hdots & a & \hdots & \frac12b & \hdots \\
                & \vdots & \ddots & \vdots & \\
       \hdots & 2c & \hdots & d & \hdots \\
              & \vdots & & \vdots &
    \end{pmatrix}.
\end{align*}
Since $T$ has the same matrix with respect to
every basis of $V,$ we see that
\begin{align*}
    b=\frac12 b\quad\text{and}\quad c=2c.
\end{align*}
Thus, $b=c=0.$
\end{proof}

\begin{claim*}
    $T$ is a scalar multiple of the identity operator.
\end{claim*}
\begin{proof}
Our first claim implies that entries on the diagonal of
$\SM(T,(v_1,\ldots,v_n))$ are all pairwise equal. Thus, they
are all equal. Our second claim implies all other entries
are zero. Thus,
\begin{align*}
    \SM(T,(v_1,\ldots,v_n))=\begin{pmatrix}
        \lambda & & 0 \\
                & \ddots & \\
        0 & & \lambda
    \end{pmatrix}
    =\SM(\lambda I)
\end{align*}
for some $\lambda\in\F.$
\end{proof}

\vspace{0.5in}

\begin{note*}
You can view the source code for this solution
\href{https://github.com/thomasbreydo/linalg/blob/main/\pagenum_\probnum_Thomas_Breydo.tex}
{here}.
\end{note*}

\end{document}
